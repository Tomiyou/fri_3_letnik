\documentclass[12pt]{article}
\usepackage[utf8]{inputenc}
\usepackage{amsmath}
\usepackage{palatino}
\usepackage{enumitem}
\usepackage{amssymb}

\title{Assignment 3}
\author{Tomaž Hribernik}
\date{December 2019}

\begin{document}

\maketitle

\section{Amortization}
	We need to calculate the amortized cost of this function.
	\begin{gather*}
	c_i =
	\begin{cases}
	i + c & ;i \neq k^{2}, k \in \mathbb N \\
	i^{2} & ;i = k^{2}, k \in \mathbb N
	\end{cases}
	\end{gather*}
	I did this using the aggregate method. Between 1 and N (both inclusive) there is $\sqrt{N}$ numbers that are perfect squares, that's what I use to sum cost when the second case is true in the function above. I also subtracted cost of the first case in the second sum as we count it $\sqrt{N}$ too many times otherwise
	\begin{gather*}
	\frac{1}{n}\sum_{i=1}^{n}{c_i} = \frac{1}{n}(\sum_{i=1}^{n}{(i + c)} + \sum_{i=1}^{\left \lfloor{\sqrt{n}}\right \rfloor}{(i^{4}-(i^2 + c))}) = \\
	\frac{1}{n}(\sum_{i=1}^{n}{i} + \sum_{i=1}^{n}{c}) + \frac{1}{n}(\sum_{i=1}^{\left \lfloor{\sqrt{n}}\right \rfloor}{i^{4}} - \sum_{i=1}^{\left \lfloor{\sqrt{n}}\right \rfloor}{(i^2 + c)})
	\end{gather*}
	I also used the following formulas
	\begin{gather*}
	\sum_{i=1}^{n}{i^{4}} = \frac{n(n+1)(6n^3 + 9n^2 + n - 1)}{30}\\	
	\sum_{i=1}^{n}{i^{2}} = \frac{n(n+1)(2n+1)}{6}
	\end{gather*}
	\begin{gather*}
	\frac{1}{n}(\sum_{i=1}^{n}{i} + \sum_{i=1}^{n}{c}) + \frac{1}{n}(\sum_{i=1}^{\left \lfloor{\sqrt{n}}\right \rfloor}{i^{4}} - \sum_{i=1}^{\left \lfloor{\sqrt{n}}\right \rfloor}{(i^2 + c)}) =\\
	\frac{n(n+1)}{2n} + \frac{nc}{n} + \frac{\sqrt{n}(\sqrt{n} + 1)(6\sqrt{n}^{3} + 9\sqrt{n}^{2} + \sqrt{n} - 1)}{30n} - \frac{1}{n}\sum_{i=1}^{\left \lfloor{\sqrt{n}}\right \rfloor}{(i^2 + c)} = \\
	\frac{n+1}{2} + c + \frac{(n + \sqrt{n})(6\sqrt{n}^{3} + 9\sqrt{n}^{2} + \sqrt{n} - 1)}{30n} - \frac{\sqrt{n}(\sqrt{n} + 1)(2\sqrt{n} + 1)}{6n} - \frac{\sqrt{n}c}{n} =\\
	\frac{n}{2} + \frac{1}{2} + c + \frac{\sqrt{n}^{3}}{5} + \frac{n}{2} + \frac{\sqrt{n}}{3} - \frac{1}{30\sqrt{n}} - \frac{(n + \sqrt{n})(2\sqrt{n} + 1)}{6n} - \frac{c}{\sqrt{n}} =\\
	n + \frac{1}{2} + c + \frac{\sqrt{n}^{3}}{5} + \frac{\sqrt{n}}{3} - \frac{1}{30\sqrt{n}} - \frac{\sqrt{n}}{3} - \frac{1}{6\sqrt{n}} - \frac{1}{2} - \frac{c}{\sqrt{n}} =\\
	n + c + \frac{\sqrt{n}^{3}}{5} - \frac{1}{30\sqrt{n}} - \frac{1}{6\sqrt{n}} - \frac{c}{\sqrt{n}}
	\end{gather*}
	Want to know the amortized cost as n approaches infinity, so we use limit to get the final result
	\begin{gather*}
		\lim_{n\to\infty}(n + c + \frac{\sqrt{n}^{3}}{5} - \frac{1}{30\sqrt{n}} - \frac{1}{6\sqrt{n}} - \frac{c}{\sqrt{n}}) = \frac{\sqrt{n}^{3}}{5} = \sqrt{n}^{3} = c_i\sp{\prime}
	\end{gather*}
	The amortized cost for this function is $c_i\sp{\prime} = \sqrt{n}^{3}$ where n is the number of calls of the function.

\section{Amortization}
	In order to prove that even when we only expand the table by 10\% amortized cost is still constant, I presumed it true, and worked my way from the end to the start while taking lectures as an example.
	\begin{gather*}
		c_i\sp{\prime} = c_i + \Phi(D_i) - \Phi(D_{i-1})\\
		\Phi(D_i) = 11 \cdot num_i - 10 \cdot size_i
	\end{gather*}
	11 and 10 were selected as coefficients as they nicely resolve the equation (when I worked from my way from the end to start) and are basically just multiplied by 10 from the original equation $\Phi(D_i) = 1,1 \cdot num_i - size_i$ from lectures. They also satisfy the condition that $\Phi(D_i) \geq 0$
	\begin{enumerate}[label=\alph*)]
		\item Expansion on i-th operation\\
		When expanding the following holds
		\begin{gather*}
			size_i = 1.1 \cdot size_{i-1}\\
			size_{i-1} = num_{i-1} = num_i - 1
		\end{gather*}
		Now we prove that amortized cost is still constant
		\begin{gather*}
			c_i\sp{\prime} = c_i + \Phi(D_i) - \Phi(D_{i-1}) =\\
			num_i + (11 \cdot num_i - 10 \cdot size_i) - (11 \cdot num_{i-1} - 10 \cdot size_{i-1}) =\\
			12num_i - 10(1.1 \cdot size_{i-1}) - 11(num_i - 1) + 10(num_i - 1) =\\
			12num_i - 11 \cdot size_{i-1} - 11num_i + 11 + 10num_i - 10 =\\
			11num_i - 11(num_i - 1) + 1 = 11num_i - 11num_i + 11 + 1 = 12
		\end{gather*}
		
		\item NO expansion on i-th operation\\
		When expanding the following holds
		\begin{gather*}
		size_i = size_{i-1}\\
		num_{i-1} = num_i - 1
		\end{gather*}
		Now we prove that amortized cost is still constant
		\begin{gather*}
		c_i\sp{\prime} = c_i + \Phi(D_i) - \Phi(D_{i-1}) =\\
		1 + (11 \cdot num_i - 10 \cdot size_i) - (11 \cdot num_{i-1} - 10 \cdot size_{i-1}) =\\
		1 + 11 \cdot num_i - 10 \cdot size_i - 11(num_i - 1) + 10 \cdot size_i =\\
		1 + 11 \cdot num_i - 11 \cdot num_i + 11 = 12
		\end{gather*}
	\end{enumerate}
\section{Approximation}
	First we need to calculate the probability of satisfying one pair of symmetric clauses
	\begin{gather*}
	P(clause_1 \land clause_2 = 1) = 1 - P(clause_1 \land clause_2 = 0)
	\end{gather*}
	There are 16 possible inputs to each for each clause, but only one where the output is 0 since each clause is in conjunctive normal form
	\begin{gather*}
	P(\textit{clause is not satisfied}) = \frac{1}{2} \cdot \frac{1}{2} \cdot \frac{1}{2} \cdot \frac{1}{2} = \frac{1}{16}
	\end{gather*}
	For $clause_1 \land clause_2$ to be 0, either has to be 0 and since they cannot be 0 at
	the same time (their terms are all negated), this means 2 of 16 possible inputs result in output being 0.
	\begin{gather*}
	P(clause_1 \land clause_2 = 0) = \frac{2}{16}\\
	P(clause_1 \land clause_2 = 1) = 1 - \frac{2}{16} = \frac{14}{16} = \frac{7}{8} \Rightarrow E(Y_i) = \frac{7}{8}
	\end{gather*}
	Now we can start finding the approximation factor for this algorithm
	\begin{gather*}
	Y = \textit{number of satisfied pairs of clauses} = Y_1 + Y_2 + \dots + Y_n \\
	E(Y) = E(Y_1 + Y_2 + \dots + Y_n) = E(\sum_{i=1}^{n}{Y_i}) = \sum_{i=1}^{n}{E(Y_i)} = \sum_{i=1}^{n}{\frac{7}{8}}\\
	E(Y) = \frac{7}{8}n = C; C^{*} = n \Rightarrow \frac{C^{*}}{C} = \frac{n}{\frac{7}{8}n} = \frac{8}{7}
	\end{gather*}

\end{document}
