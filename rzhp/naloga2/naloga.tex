\documentclass[12pt]{article}
\usepackage[utf8]{inputenc}
\usepackage{amsmath}
\usepackage{palatino}
\usepackage{enumitem}

\title{Assignment 2}
\author{Tomaž Hribernik}
\date{November 2019}

\begin{document}

\maketitle

\section{Drop sort}
	First we define the indicator variable
	\begin{gather*}
	X_i = \textit{event where we keep the i-th item in the array}
	\end{gather*}
		In a uniformly randomly distributed array, each sub-array i also uniformly randomly distributed. That means that the probability of the last element being largest in a sub-array of size x, is $\frac{1}{x}$, as each place in the array has an equal probability of having the largest element. This means that we keep the i-th element if it is biggest element up to this point $\Rightarrow$
	\begin{gather*}
	P(X_i) = \frac{1}{i} \Rightarrow E(X_i) = \frac{1}{i} \textit{, where i is the position in the array}
	\end{gather*}
	Now we can calculate the expected value of n elements, adjusting for the first two, since we always keep them. We can also use the same way of calculation for both ascending and descending since the probabilities do not change
	\begin{gather*}
		E(X) = 2 + E(X_3 + X_4 + \dots + X_n) = 2 + E(\sum_{i=3}^{n}{X_i}) = 2 + \sum_{i=3}^{n}{E(X_i)}\\
		E(X) = 2 + \sum_{i=1}^{n}{E(X_i)} - \sum_{i=1}^{2}{E(X_i)} = 2 + \sum_{i=1}^{n}{\frac{1}{i}} - \sum_{i=1}^{2}{\frac{1}{i}}\\
		E(X) = 2 + H_n - \frac{1}{1} - \frac{1}{2} = \frac{1}{2} + H_n \approx 0.5 + ln(n)
	\end{gather*}

\section{Bloom filter}
\begin{enumerate}[label=\alph*)]
\item How many bits are set on 1?\\
	Note: in both a) and b) I assume that each insert is independent\\
	First we define the indicator variable
	\begin{gather*}
	X_i = \textit{event where a specific bit is set to 0 after n inserts} \\
	1 - \frac{1}{n} = \textit{probability of a bit being 0 after 1 insert of 1 hash function}\\
	(1 - \frac{1}{n})^k = \textit{probability of a bit being 0 after 1 insert of k hash functions}\Rightarrow\\
	P(X_i) = ((1 - \frac{1}{n})^k)^n = \textit{probability of a bit being 0 after n inserts of k hash}\\
	\textit{functions} = ((1 - \frac{1}{n})^n)^k = e^{-k} \Rightarrow E(X_i) = e^{-k}
	\end{gather*}
	Now calculate the expected number of 0 bits in the array using the indicator variable
	\begin{gather*}
		E(X) = E(X_1 + X_2 + \dots + X_n) = E(\sum_{i=1}^{n}{X_i}) = \sum_{i=1}^{n}{E(X_i)}\\
		E(X) = \sum_{i=1}^{n}{E(X_i)} = \sum_{i=1}^{n}{e^{-k}} = ne^{-k} = ne^{-5} = \textit{expected number of 0 bits}\\
		E(\textit{number of 1 bits}) = \textit{total number of bits} - \textit{number of 0 bits} = n - ne^{-5}\\
		E(\textit{number of 1 bits}) = n(1 - e^{-5}) \approx 0.993262n
	\end{gather*}

\item What is the probability of a false positive?\\
	We already calculated the probability of any bit being 0
	\begin{gather*}
	P(\textit{a bit is 0 after n insertions}) = e^{-k} \Rightarrow\\
	P(\textit{a bit is 1 after n insertions}) = 1 - e^{-k}
	\end{gather*}
	For a false positive to occur all k bits checked by hash functions must be 1
	\begin{gather*}
	P(\textit{false positive}) = (P(\textit{a bit is 1 after n insertions}))^k = (1 - e^{-k})^k =\\
	(1 - e^{-5})^5 \approx 0.966761215
	\end{gather*}
\end{enumerate}

\section{Sticker album}
	First we define the indicator variable
	\begin{gather*}
	X_i = \textit{event where we get a sticker we don't already have} \sim G(p) \\
	P(X_i) = \frac{n-i}{n}\textit{, where i is the number of stickers we already own}
	\end{gather*}
	As the indicator variable $X_i$ is distributed geometrically, it's expected value is 
	\begin{equation*}
		E(X_i) = \frac{1}{p} = \frac{n}{n-i}
	\end{equation*}
	Now we can calculate the expected value for n stickers, as n repeats of $X_i$ gives us exactly a complete set of stickers
	\begin{gather*}
		E(X) = E(X_0 + X_1 + \dots + X_{n-1}) = E(\sum_{i=0}^{n-1}{X_i}) = \sum_{i=0}^{n-1}{E(X_i)}\\
		E(X) = \sum_{i=0}^{n-1}{E(X_i)} = \sum_{i=0}^{n-1}{\frac{n}{n-i}} = 
		n\sum_{i=0}^{n-1}{\frac{1}{n-i}} = n(\frac{1}{n} + \frac{1}{n-1} + \dots + \frac{1}{2} + \frac{1}{1})\\
		E(X) = n(\frac{1}{1} + \frac{1}{2} + \dots + \frac{1}{n-1} + \frac{1}{n}) = n * H_n \approx n * ln(n)
	\end{gather*}

\end{document}
